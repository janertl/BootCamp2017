\documentclass[letterpaper,12pt]{article}

\usepackage{amsmath, amsfonts, amscd, amssymb, amsthm}
\usepackage{graphicx}
%\usepackage{import}
\usepackage{versions}
\usepackage{crop}
\usepackage{multicol}
\usepackage{graphicx}
\usepackage{makeidx}
\usepackage{hyperref}
\usepackage{ifthen}
\usepackage[format=hang,font=normalsize,labelfont=bf]{caption}
\usepackage{natbib}
\usepackage{setspace}
\usepackage{placeins}
\usepackage{framed}
\usepackage{enumitem}
\usepackage{threeparttable}
\usepackage{geometry}
\geometry{letterpaper,tmargin=1in,bmargin=1in,lmargin=1in,rmargin=1in}
\usepackage{multirow}
\usepackage[table]{xcolor}
\usepackage{array}
\usepackage{delarray}
\usepackage{lscape}
\usepackage{float,color, colortbl}
%\usepackage[pdftex]{graphicx}
\usepackage{hyperref}
\usepackage{tabu}
\usepackage{appendix}
\usepackage{listings}


\include{thmstyle}
\bibliographystyle{aer}
\providecommand{\abs}[1]{\lvert#1\rvert}
\providecommand{\norm}[1]{\lVert#1\rVert}
\newcommand{\ve}{\varepsilon}
\newcommand{\ip}[2]{\langle #1,#2 \rangle}

\hypersetup{colorlinks,linkcolor=red,urlcolor=blue,citecolor=red}
\theoremstyle{definition}
\newtheorem{theorem}{Theorem}
\newtheorem{acknowledgement}[theorem]{Acknowledgement}
\newtheorem{algorithm}[theorem]{Algorithm}
\newtheorem{axiom}[theorem]{Axiom}
\newtheorem{case}[theorem]{Case}
\newtheorem{claim}[theorem]{Claim}
\newtheorem{conclusion}[theorem]{Conclusion}
\newtheorem{condition}[theorem]{Condition}
\newtheorem{conjecture}[theorem]{Conjecture}
\newtheorem{corollary}[theorem]{Corollary}
\newtheorem{criterion}[theorem]{Criterion}
\newtheorem{definition}{Definition} % Number definitions on their own
\newtheorem{derivation}{Derivation} % Number derivations on their own
\newtheorem{example}[theorem]{Example}
\newtheorem{exercise}[theorem]{Exercise}
\newtheorem{lemma}[theorem]{Lemma}
\newtheorem{notation}[theorem]{Notation}
\newtheorem{problem}[theorem]{Problem}
\newtheorem{proposition}{Proposition} % Number propositions on their own
\newtheorem{remark}[theorem]{Remark}
\newtheorem{solution}[theorem]{Solution}
\newtheorem{summary}[theorem]{Summary}
%\numberwithin{equation}{document}
\graphicspath{{./Figures/}}
\renewcommand\theenumi{\roman{enumi}}
\DeclareMathOperator*{\argmin}{arg\,min}

\crop
\makeindex


\begin{document}

\begin{titlepage}
	\title{Open Source Macroeconomics Laboratory Boot Camp \\ DSGE Exercises}  %change title here accordingly
	\author{Jan Ertl\\ \emph{University of Chicago}}
	\date{\LARGE{2017}}
	\maketitle
\end{titlepage}

\begin{spacing}{1.5}


\section*{DSGE}\label{DSGE_HW}

	% 1
	\begin{exercise} \label{DSGE_HW_BM_FindA}
		For the Brock and Mirman model, find the value of $A$ in the policy function.  Show that your value is correct.

		For this case find an algebraic solution for the policy function, $k_{t+1} = \Phi (k_t,z_t)$.  Couple of good sources for hints are \citet[exercise 2.2, p. 12]{StokeyLucas1989} and \citet[exercise 1.1, p. 47]{Sargent1987}.
	\end{exercise}

\paragraph{Solution} As suggested, we guess that the policy function is $k_{t+1} = \Phi (k_t,z_t) 
= Ae^{z_t}k_t^{\alpha}$, for some A.

To verify and find such A, we substitute into both sides of the Euler equation
$$ \frac{1}{e^{z_t}k_t^\alpha-k_{t+1}} = \beta E_t\{\frac{\alpha e^{z_{t+1}} k_{t+1}^{\alpha-1}}{e^{z_{t+1}}k_{t+1}^\alpha-k_{t+2}}\}$$

We get 

$$ \frac{1}{e^{z_t}k_t^{\alpha}(1-A)} = \frac{1}{e^{z_t}k_t^\alpha-Ae^{z_t}k_t^{\alpha}} $$
$$= \beta E_t\{\frac{\alpha e^{z_{t+1}} k_{t+1}^{\alpha-1}}{e^{z_{t+1}}k_{t+1}^\alpha-k_{t+2}}\} = \beta(\frac{\alpha e^{\rho z_t}(Ae^{z_t}k_t^\alpha)^{\alpha-1}}{e^{\rho z_t}(Ae^{z_t}k_t^\alpha)^{\alpha}(1-A)})
= \frac{\alpha\beta}{Ae^{z_t}k_t^\alpha(1-A)}$$

which simplifies without contradiction to $1 = \frac{\alpha \beta}{A}$, i.e. $A = \alpha \beta$.


	% 2
	\begin{exercise} \label{DSGE_HW_CharEq_Ln}
		For the model in section 3 of the notes consider the following functional forms:
		\begin{equation}\label{DSGE_HW_CharEq_Ln_eq01}
		\begin{split}
		u(c_t,\ell_t) & = \text{ln }c_t + a \text{ ln }(1-\ell_t)\\
		F(K_t,L_t,z_t) & = e^{z_t}K^{\alpha}_t L^{1-\alpha}_t  \nonumber
		\end{split}
		\end{equation}
		Write out all the characterizing equations for the model using these functional forms.

		Can you use the same tricks as in Exercise 1 to solve for the policy function in this case?  Why or why not?
	\end{exercise}

\paragraph{Solution} Regardless of functional form, we always have the following market clearing conditions
$$\ell_t = L_t, \quad k_t = K_t, \quad w_t = W_t, \quad r_t = R_t ,$$

as well as the government budget balance and law of motion of the shock

$$ \tau [w_tl_t + (r_t - \delta)k_t] = T_t, \quad z_t = ( 1 - \rho_z )\overline{z} + \rho_z z_{t - 1} + \epsilon_t^z; \quad \epsilon_t^z \sim i.i.d.(0, \sigma_z^2)$$

The remaining characterizing equations with the above-mentioned functional form:

$$c_t = (1 - \tau)[w_t\ell_t + (r_t - \delta) k_t] + k_t + T_t - k_{t + 1} $$
$$\frac{1}{c_t} = \beta E_t{\frac{1}{c_{t + 1}} [ (r_{t + 1} -\delta)(1 - \tau) + 1]} $$
$$ \frac{a}{1 - \ell_t} = \frac{1}{c_t} w_t (1 - \tau) $$
$$r_t = \alpha e^{z_t} (\frac{ \ell_t}{k_t})^{1 - \alpha} $$
$$w_t = (1 - \alpha) e^{z_t} (\frac{k_t}{\ell_t})^\alpha$$

	% 3
	\begin{exercise} \label{DSGE_HW_CharEq_CES_Ln}
		For the model in section 3 consider the following functional forms:
		\begin{equation}\label{DSGE_HW_CharEq_CES_Ln_eq01}
		\begin{split}
		u(c_t,\ell_t) & = \frac{c^{1-\gamma}_t -1}{1-\gamma}+ a \text{ ln }(1-\ell_t)\\
		F(K_t,L_t,z_t) & = e^{z_t}K^{\alpha}_t L^{1-\alpha}_t  \nonumber
		\end{split}
		\end{equation}
		Write out all the characterizing equations for the model using these functional forms.
	\end{exercise}

\paragraph{Solution} Regardless of functional form, we always have the following market clearing conditions
$$\ell_t = L_t, \quad k_t = K_t, \quad w_t = W_t, \quad r_t = R_t ,$$

as well as the government budget balance and law of motion of the shock

$$ \tau [w_tl_t + (r_t - \delta)k_t] = T_t, \quad z_t = ( 1 - \rho_z )\overline{z} + \rho_z z_{t - 1} + \epsilon_t^z; \quad \epsilon_t^z \sim i.i.d.(0, \sigma_z^2)$$

The remaining characterizing equations with the above-mentioned functional form:

$$c_t = (1 - \tau)[w_t\ell_t + (r_t - \delta) k_t] + k_t + T_t - k_{t + 1} $$
$$\frac{1}{c_t^\gamma} = \beta E_t{\frac{1}{c_{t + 1}^\gamma} [ (r_{t + 1} -\delta)(1 - \tau) + 1]} $$
$$ \frac{a}{1 - \ell_t} = \frac{1}{c_t^\gamma} w_t (1 - \tau) $$
$$r_t = \alpha e^{z_t} (\frac{ \ell_t}{k_t})^{1 - \alpha} $$
$$w_t = (1 - \alpha) e^{z_t} (\frac{k_t}{\ell_t})^\alpha$$



	% 4
	\begin{exercise} \label{DSGE_HW_CharEq_CES}
		For the model in section 3 consider the following functional forms:
		\begin{equation}\label{DSGE_HW_CharEq_CES_eq01}
		\begin{split}
		u(c_t,\ell_t) & = \frac{c^{1-\gamma}_t -1}{1-\gamma}+ a \frac{(1-\ell_t)^{1-\xi}-1}{1-\xi}      \\
		F(K_t,L_t,z_t) & = e^{z_t}\left[\alpha K^{\eta}_t +(1-\alpha)L^{\eta}_t \right]^{\frac{1}{\eta}}   \nonumber
		\end{split}
		\end{equation}
		Write out all the characterizing equations for the model using these functional forms.
	\end{exercise}

\paragraph{Solution} Regardless of functional form, we always have the following market clearing conditions
$$\ell_t = L_t, \quad k_t = K_t, \quad w_t = W_t, \quad r_t = R_t ,$$

as well as the government budget balance and law of motion of the shock

$$ \tau [w_tl_t + (r_t - \delta)k_t] = T_t, \quad z_t = ( 1 - \rho_z )\overline{z} + \rho_z z_{t - 1} + \epsilon_t^z; \quad \epsilon_t^z \sim i.i.d.(0, \sigma_z^2)$$

The remaining characterizing equations with the above-mentioned functional form:

$$c_t = (1 - \tau)[w_t\ell_t + (r_t - \delta) k_t] + k_t + T_t - k_{t + 1} $$
$$\frac{1}{c_t^\gamma} = \beta E_t{\frac{1}{c_{t + 1}^\gamma} [ (r_{t + 1} -\delta)(1 - \tau) + 1]} $$
$$ \frac{a}{(1 - \ell_t)^\xi} = \frac{1}{c_t^\gamma} w_t (1 - \tau) $$
$$ r_t = \alpha k_t^{\eta-1} e^{z_t}[\alpha k_t^{\eta}+(1-\alpha)\ell_t^\eta]^{\frac{1}{\eta}-1}$$ 
$$w_t = (1 - \alpha) \ell_t^{\eta-1} e^{z_t}[\alpha k_t^{\eta}+(1-\alpha)\ell_t^\eta]^{\frac{1}{\eta}-1}$$




	% 5
	\begin{exercise} \label{DSGE_HW_NoLeisure}
		For the model in section 3 abstract from the labor/leisure decision and consider the following functional forms:
		\begin{equation}\label{DSGE_HW_NoLeisure_eq01}
		\begin{split}
		u(c_t) & = \frac{c^{1-\gamma}_t -1}{1-\gamma}      \\
		F(K_t,L_t,z_t) & = K^{\alpha}_t (L_te^{z_t})^{1-\alpha}  \nonumber
		\end{split}
		\end{equation}
		Write out all the characterizing equations for the model using these functional forms.  Assume $\ell_t=1$.

		Write out the steady state versions of these equations.  Solve algebraically for the steady state value of $k$ as a function of the steady state value of $z$ and the parameters of the model.  Numerically solve for the steady state values of all variables using the following parameter values: $\gamma = 2.5$, $\beta = .98$, $\alpha = .40$, $\delta = .10$, $\bar z = 0$ and $\tau = .05$.  Also solve for the steady state values of output and investment.  Compare these values with the ones implied by the algebraic solution.
	\end{exercise}

\paragraph{Solution} Regardless of functional form, we always have the following market clearing conditions
$$\ell_t = L_t, \quad k_t = K_t, \quad w_t = W_t, \quad r_t = R_t ,$$

as well as the government budget balance and law of motion of the shock

$$ \tau [w_tl_t + (r_t - \delta)k_t] = T_t, \quad z_t = ( 1 - \rho_z )\overline{z} + \rho_z z_{t - 1} + \epsilon_t^z; \quad \epsilon_t^z \sim i.i.d.(0, \sigma_z^2)$$

The remaining characterizing equations with the above-mentioned functional form:

$$c_t = (1 - \tau)[w_t\ell_t + (r_t - \delta) k_t] + k_t + T_t - k_{t + 1} $$
$$\frac{1}{c_t^\gamma} = \beta E_t{\frac{1}{c_{t + 1}^\gamma} [ (r_{t + 1} -\delta)(1 - \tau) + 1]} $$
$$ \ell_t = 1, \quad \text{which makes} \quad 0 = u_{\ell} = \frac{1}{c_t^\gamma} w_t (1 - \tau) \quad \text{unnecessary}$$
$$r_t = \alpha (\frac{ e^{z_t} \ell_t}{k_t})^{1 - \alpha} $$
$$ w_t = (1 - \alpha) e^{z_t} (\frac{k_t}{e^{z_t}})^\alpha$$

In the steady state, note that through $e^{\bar{z}} = e^0 = 1$ and budget balance, these become:

$$\bar{c} = \bar{w} + (\bar{r} - \delta) \bar{k}$$
$$\frac{1}{\bar{c}^\gamma} = \beta {\frac{1}{\bar{c}^\gamma} [(\bar{r} -\delta)(1 - \tau) + 1]} $$
$$\bar{r} = \alpha (\frac{1}{\bar{k}})^{1 - \alpha} $$
$$\bar{w} = (1 - \alpha) (\bar{k})^\alpha$$

We can cancel $c$ out of the Euler equation, and immediately solve for $\bar{r}$:

$$\bar{r} = \delta + \frac{1 - \beta}{\beta(1 -\tau)} = \frac{1 - \beta + \delta \beta(1 -\tau)}{\beta(1 -\tau)} $$
We can then use the third equation to solve for $\bar{k}$:

$$\bar{k} = (\frac{\alpha}{\bar{r}} )^{\frac{1}{1 - \alpha}} =(\frac{\alpha \beta(1 -\tau)}{1 - \beta + \delta \beta(1 -\tau)} )^{\frac{1}{1 - \alpha}} $$

and the fourth equation to yield wage:

$$\bar{w} = (1 - \alpha) \bar{k}^{\alpha} = (1 - \alpha)(\frac{\alpha \beta(1 -\tau)}{1 - \beta + \delta \beta(1 -\tau)} )^{\frac{\alpha}{1 - \alpha}} $$

Likewise, the remaining variables can be easily computed through substitution:

$$ \bar{c} = \bar{w} + (\bar{r} - \delta) \bar{k}, \qquad 
\bar{Y} = \bar{k}^\alpha, \qquad 
\bar{I} = \delta \bar{k} $$

See the notebook for the calculations.


	% 6
	\begin{exercise} \label{DSGE_HW_CES}
		For the model in section 3 consider the following functional forms:
		\begin{equation}\label{DSGE_HW_CES_eq01}
		\begin{split}
		u(c_t,\ell_t) & = \frac{c^{1-\gamma}_t -1}{1-\gamma}+ a \frac{(1-\ell_t)^{1-\xi}-1}{1-\xi}      \\
		F(K_t,L_t,z_t) & = K^{\alpha}_t (L_te^{z_t})^{1-\alpha}  \nonumber
		\end{split}
		\end{equation}
		Write out all the characterizing equations for the model using these functional forms.  {}Write out the steady state versions of these equations.  Numerically solve for the steady state values of all variables using the following parameter values: $\gamma = 2.5$, $\xi = 1.5$,  $\beta = .98$, $\alpha = .40$, $a=.5$, $\delta = .10$, $\bar z = 0$, and $\tau = .05$.  Also solve for the steady state values of output and investment.
	\end{exercise}

\paragraph{Solution} Regardless of functional form, we always have the following market clearing conditions
$$\ell_t = L_t, \quad k_t = K_t, \quad w_t = W_t, \quad r_t = R_t ,$$

as well as the government budget balance and law of motion of the shock

$$ \tau [w_tl_t + (r_t - \delta)k_t] = T_t, \quad z_t = ( 1 - \rho_z )\overline{z} + \rho_z z_{t - 1} + \epsilon_t^z; \quad \epsilon_t^z \sim i.i.d.(0, \sigma_z^2)$$

The remaining characterizing equations with the above-mentioned functional form:

$$c_t = (1 - \tau)[w_t\ell_t + (r_t - \delta) k_t] + k_t + T_t - k_{t + 1} $$
$$\frac{1}{c_t^\gamma} = \beta E_t{\frac{1}{c_{t + 1}^\gamma} [ (r_{t + 1} -\delta)(1 - \tau) + 1]} $$
$$ \frac{a}{(1 - \ell_t)^\xi} = \frac{1}{c_t^\gamma} w_t (1 - \tau) $$
$$r_t = \alpha e^{z_t} (\frac{ e^{z_t} \ell_t}{k_t})^{1 - \alpha} $$
$$w_t = (1 - \alpha) e^{z_t} (\frac{k_t}{e^{z_t} \ell_t})^\alpha$$

In the steady state:


$$\bar{c} = \bar{w} + (\bar{r}\bar{\ell} - \delta) \bar{k}$$
$$\frac{1}{\bar{c}^\gamma} = \beta {\frac{1}{\bar{c}^\gamma} [(\bar{r} -\delta)(1 - \tau) + 1]} $$
$$ \frac{a}{(1 - \bar{\ell})^\xi} = \frac{1}{\bar{c}^\gamma} \bar{w} (1 - \tau) $$
$$\bar{r} = \alpha e^{\bar{z}} (\frac{\bar{\ell}e^{\bar{z}}}{\bar{k}})^{1 - \alpha} $$
$$\bar{w} = (1 - \alpha) (\frac{\bar{k}}{\bar{\ell}e^{\bar{z}}})^\alpha$$

The remaining exercises can be found in the notebook.

	% 7
	%\begin{exercise} \label{DSGE_HW_Base_TotalDiff}
	%	For the steady state of the baseline tax model in section \ref{DSGE_SS_Base} use numerical differentiation to solve for the full set of comparative statics and sign them where possible.  Find $\frac{\partial y}{\partial x}$ for $y\in\{\bar k, \bar \ell, \bar y, \bar w, \bar r, \bar T, \bar i, \bar c \}$ and $x\in\{\alpha, \beta, \gamma, \delta, \xi, \tau, a, \bar z\}$.

	%	Using the same parameter values as in excercise \ref{DSGE_HW_CES}, solve for the numerical values of the comparative statics.
	%\end{exercise}

%\paragraph{Solution}

	%8
	%\begin{exercise} \label{DSGE_HW_BM_Grid}
	%	For the Brock and Mirman model in section \eqref{DSGE_BrockMirman} set up a discrete grid for $K$ with 25 values ranging from $.5 \bar K$ to $1.5 \bar K$.  Also set up a discrete grid for $z$ with 25 values ranging from $-5\sigma$ to $+5\sigma$.  Set up a value function array, $V$ that stores the value for all $26^2$ possible permutations of $K$ and $z$.  Also set up a policy function array, $H$, that stores the optimal index value of $K'$ for all all $25^2$ possible permutations of $K$ and $z$.

	%	To begin assume that all elements of $V$ are zero and that all elements of $H$ point to the lowest possible value for $K$ ($.5 \bar K$).

	%	Loop over all possible values of $K$ and $z$ and for each combination find 1) the optimal value of $K'$ from the 25 possible values.  Store this value in an updated policy function array, $H_{new}$.  Also find 2) the value implied by this choice given the current value function.  Store this in an updated value function array, $V_{new}$.

	%	Once this is completed for all $K$ and $z$ check to see if $V$ is approximately equal to $V_{new}$.  If so, output the value function and policy function arrays.  If not, replace $V$ with $V_{new}$ and $H$ with $H_{new}$ and repeat the search above.

	%	When finished plot the three-dimensional surface plot for the policy function $K' = H(K,z)$.  Compare this with the closed form solution as described in section \ref{DSGE_BrockMirman}.
	%\end{exercise}

%\paragraph{Solution}



\end{spacing}

\newpage

\bibliography{BootCampText}

\end{document}
