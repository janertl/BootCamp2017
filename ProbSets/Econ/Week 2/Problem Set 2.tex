\documentclass[letterpaper,12pt]{article}
\usepackage{array}
\usepackage{threeparttable}
\usepackage{geometry}
\usepackage{mathrsfs}
\usepackage{dsfont}
\geometry{letterpaper,tmargin=1in,bmargin=1in,lmargin=1.25in,rmargin=1.25in}
\usepackage{fancyhdr,lastpage}
\usepackage{setspace}
\usepackage{enumitem}
\usepackage{gensymb}
\pagestyle{fancy}
\lhead{}
\chead{}
\rhead{}
\lfoot{\footnotesize\textsl{OSM Lab, Summer 2017, ECON PS \#2}}
\cfoot{}
\rfoot{\footnotesize\textsl{Page \thepage\ of \pageref{LastPage}}}
\renewcommand\headrulewidth{0pt}
\renewcommand\footrulewidth{0pt}
\usepackage[format=hang,font=normalsize,labelfont=bf]{caption}
\usepackage{amsmath}
\usepackage{amssymb}
\usepackage{amsthm}
\usepackage{natbib}
\usepackage{setspace}
\usepackage{float,color}
\usepackage[pdftex]{graphicx}
\usepackage{hyperref}
\hypersetup{colorlinks,linkcolor=red,urlcolor=blue,citecolor=red}
\theoremstyle{definition}
\newtheorem{theorem}{Theorem}
\newtheorem{acknowledgement}[theorem]{Acknowledgement}
\newtheorem{algorithm}[theorem]{Algorithm}
\newtheorem{axiom}[theorem]{Axiom}
\newtheorem{case}[theorem]{Case}
\newtheorem{claim}[theorem]{Claim}
\newtheorem{conclusion}[theorem]{Conclusion}
\newtheorem{condition}[theorem]{Condition}
\newtheorem{conjecture}[theorem]{Conjecture}
\newtheorem{corollary}[theorem]{Corollary}
\newtheorem{criterion}[theorem]{Criterion}
\newtheorem{definition}[theorem]{Definition}
\newtheorem{derivation}{Derivation} % Number derivations on their own
\newtheorem{example}[theorem]{Example}
\newtheorem{exercise}[theorem]{Exercise}
\newtheorem{lemma}[theorem]{Lemma}
\newtheorem{notation}[theorem]{Notation}
\newtheorem{problem}[theorem]{Problem}
\newtheorem{proposition}{Proposition} % Number propositions on their own
\newtheorem{remark}[theorem]{Remark}
\newtheorem{solution}[theorem]{Solution}
\newtheorem{summary}[theorem]{Summary}
%\numberwithin{equation}{section}
\bibliographystyle{aer}
\newcommand\ve{\varepsilon}
\newcommand\boldline{\arrayrulewidth{1pt}\hline}

\begin{document}

\begin{flushleft}
   \textbf{\large{Econ, Problem Set \#2}} \\[5pt]
   OSM Lab, John Stachurski \\[5pt]
   Due Wednesday, July 5 at 8:00am
\end{flushleft}

\vspace{5mm}
   
\paragraph{Exercise 1, Problem 2}
To prove existence, we show that $ T = c (1 - \beta) + \beta
    \sum_{k=1}^K \max \left\{
        w_k ,\, x
    \right\}
    \, p_k$ 
is a contraction mapping. Take any $x, y \in \mathbb R$. Then

$$ |Tx - Ty| 
= | \beta \sum_{k=1}^K (\max\{w_k , x\} - \max\{w_k , y\}) p_k |
\le \beta \sum_{k=1}^K |(\max\{w_k , x\} - \max\{w_k , y\})| p_k$$
$$ \le \beta \sum_{k=1}^K |x - y| p_k
= \beta |x - y| \sum_{k=1}^K  p_k
= \beta |x - y|
< |x - y|
$$
So T is a contraction mapping on a complete space, ($[0, \infty]$) and hence converges to a unique fixed point. $\square$

\paragraph{Exercise 2, Problem 1}
Again, we have an operator U, $$ Uw(y) = u(\sigma(y)) + \beta \int w(f(y - \sigma(y))z) \phi(dz) \qquad (y \in \mathbb R_+)$$
and show that U is a contraction mapping by observing:

$$ \|Uw(y) - Uw'(y)\| = \|\beta \int (w(f(y - \sigma(y))z) - w'(f(y - \sigma(y))z))  \phi(dz)\| $$
$$ \le \beta \int \|w(f(y - \sigma(y))z) - w'(f(y - \sigma(y))z)\|  \phi(dz) $$
$$ \le \beta \int sup|w(f(y - \sigma(y))z) - w'(f(y - \sigma(y))z)|  \phi(dz) $$
$$  \le \beta \int |w(y) - w'(y)|  \phi(dz) 
= \beta  sup|w(y) - w'(y)| \int  \phi(dz)$$
$$ = \beta  sup|w(y) - w'(y)|
= \beta  \|w(y) - w'(y)\|
< \|w(y) - w'(y)\|$$
Once again, observing our space of $C(\mathbb{R}^+)$ is complete under the sup norm, by Banach's fixed point theorem, there exists a unique fixed point. $\square$

\paragraph{(b) $v_{\sigma}(y) =
\mathbb E \left[ \sum_{t = 0}^{\infty} \beta^t u(\sigma(y_t)) \right]$ is the fixed point.}

$$ Uv_{\sigma}(y) = u(\sigma(y)) + \beta \int v_{\sigma}(f(y - \sigma(y))z) \phi(dz) \qquad (y \in \mathbb R_+)$$
$$ = u(\sigma(y)) + \beta \mathbb{E} \left[ v_{\sigma}(y') \right]
= u(\sigma(y)) + \beta \mathbb{E} \left[ \mathbb E \left[ \sum_{t = 0}^{\infty} \beta^t u(\sigma(y_{t+1})) \right] \right]$$ 
$$ = u(\sigma(y)) + \mathbb E \left[ \sum_{t = 0}^{\infty} \beta^{t + 1} u(\sigma(y_{t+1})) \right] 
= u(\sigma(y)) + \mathbb E \left[ \sum_{t = 1}^{\infty} \beta^{t} u(\sigma(y_t)) \right] $$
$$ = \beta^0 u(\sigma(y)) + \mathbb E \left[ \sum_{t = 1}^{\infty} \beta^{t} u(\sigma(y_t)) \right] 
= \mathbb E \left[ \sum_{t = 0}^{\infty} \beta^t u(\sigma(y_t)) \right] =  v_{\sigma}(y)
\qquad (y \in \mathbb R_+)$$

where the time indices on y are necessarily shifted to start in period 1 in the second line since it represents the continuation value.

Thus, we have shown that $
v_{\sigma}(y) =
\mathbb E \left[ \sum_{t = 0}^{\infty} \beta^t u(\sigma(y_t)) \right]$ is the fixed point. $\square$

\vspace{25mm}

\bibliography{ProbStat_probset}

\end{document}