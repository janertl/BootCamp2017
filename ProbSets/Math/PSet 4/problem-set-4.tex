\documentclass[letterpaper,12pt]{article}
\usepackage{array}
\usepackage{threeparttable}
\usepackage{geometry}
\usepackage{mathrsfs}
\usepackage{dsfont}
\geometry{letterpaper,tmargin=1in,bmargin=1in,lmargin=1.25in,rmargin=1.25in}
\usepackage{fancyhdr,lastpage}
\usepackage{setspace}
\usepackage{enumitem}
\usepackage{gensymb}
\pagestyle{fancy}
\lhead{}
\chead{}
\rhead{}
\lfoot{\footnotesize\textsl{OSM Lab, Summer 2017, Math PS \#4}}
\cfoot{}
\rfoot{\footnotesize\textsl{Page \thepage\ of \pageref{LastPage}}}
\renewcommand\headrulewidth{0pt}
\renewcommand\footrulewidth{0pt}
\usepackage[format=hang,font=normalsize,labelfont=bf]{caption}
\usepackage{amsmath}
\usepackage{amssymb}
\usepackage{amsthm}
\usepackage{natbib}
\usepackage{setspace}
\usepackage{float,color}
\usepackage[pdftex]{graphicx}
\usepackage{hyperref}
\hypersetup{colorlinks,linkcolor=red,urlcolor=blue,citecolor=red}
\theoremstyle{definition}
\newtheorem{theorem}{Theorem}
\newtheorem{acknowledgement}[theorem]{Acknowledgement}
\newtheorem{algorithm}[theorem]{Algorithm}
\newtheorem{axiom}[theorem]{Axiom}
\newtheorem{case}[theorem]{Case}
\newtheorem{claim}[theorem]{Claim}
\newtheorem{conclusion}[theorem]{Conclusion}
\newtheorem{condition}[theorem]{Condition}
\newtheorem{conjecture}[theorem]{Conjecture}
\newtheorem{corollary}[theorem]{Corollary}
\newtheorem{criterion}[theorem]{Criterion}
\newtheorem{definition}[theorem]{Definition}
\newtheorem{derivation}{Derivation} % Number derivations on their own
\newtheorem{example}[theorem]{Example}
\newtheorem{exercise}[theorem]{Exercise}
\newtheorem{lemma}[theorem]{Lemma}
\newtheorem{notation}[theorem]{Notation}
\newtheorem{problem}[theorem]{Problem}
\newtheorem{proposition}{Proposition} % Number propositions on their own
\newtheorem{remark}[theorem]{Remark}
\newtheorem{solution}[theorem]{Solution}
\newtheorem{summary}[theorem]{Summary}
%\numberwithin{equation}{section}
\bibliographystyle{aer}
\newcommand\ve{\varepsilon}
\newcommand\boldline{\arrayrulewidth{1pt}\hline}

\begin{document}

\begin{flushleft}
   \textbf{\large{Math, Problem Set \#4, Optimization Introduction}} \\[5pt]
   OSM Lab, Dr. Barro\\[5pt]
   Due Friday, July 14 at 8:00am
\end{flushleft}

\vspace{5mm}
   

\paragraph{6.1} Put the following optimization problem in standard form; that is, write an optimization problem in standard form that is equivalent to the following. Given $x, y \in \mathbb{R}^n$, $a, b \in \mathbb{R}$, and $A \in M_n(\mathbb{R})$, choose $w \in \mathbb{R}^n$ in order to

$$ \text{maximize} \quad e^{-w^T x}$$
$$ \text{subject to} \quad w^T x \ge w^T A w - w^T A y + a $$
$$ \quad y^T w = w^T x + b $$

\paragraph{Solution} In the standard form (see Definition 6.1.8), this is written as 


$$ \text{minimize} \quad -e^{-w^T x}$$
$$ \text{subject to} \quad  w^T A w - w^T A y - w^T x  \le -a $$
$$ \quad y^T w - w^T x =  b $$

\paragraph{6.5} A plastics company makes two products: knobs for electronic products and milk cartons. The primary production expenses for each are labor and the raw plastic.
Each milk bottle requires 4 grams of plastic and 2 minutes of labor. Each knob takes 3 grams of plastic and 1 minute of labor. During the current production period, the company has 240kg of plastic and 100 hours of labor. Each milk bottle yields a profit of \$0.07 and each knob \$ 0.05. Write an optimization problem in standard form that is equivalent to finding the amount the company should produce of each product in order to maximize its profits.

\paragraph{Solution} We use shorthand m for milk bottles, k for knobs. We have a plastic constraint (in grams) $4 m + 3 k \le 240000 $, and time constraint (in hours) $2 m + k \le 6000 $.

The profit for the firm is given by  $0.07 m + 0.05 k$.

Writing these constraints in standard form, we get the maximization problem in standard form:

$$ \underset{m, k}{max} \quad 0.07 m + 0.05 k $$
$$ \text{such that} \quad 
\begin{bmatrix}
4 & 3 \\
2 & 1 \\
-1 & 0 \\
0 & -1 
\end{bmatrix}
\begin{bmatrix}
m \\
k
\end{bmatrix}
\le 
\begin{bmatrix}
240000 \\
6000 \\
0 \\
0
\end{bmatrix},
$$
where the inequalities are element-wise. 
                                   
\paragraph{6.6} Find and identify all the critical points of the function
$$f(x, y) = 3x^2y + 4xy^2 + xy$$
Determine whether they are the locations of local maxima, minima, or saddle points.

\paragraph{Solution} We have the derivatives:

$$f_x(x, y) = 6xy + 4y^2 + y = (6x + 4y + 1)y$$
$$f_y(x, y) = 3x^2 + 8xy + x = (3x + 8y + 1)x$$

So 

$$ Df(x,y) = \begin{bmatrix}
(6x + 4y + 1)y & x(3x + 8y + 1)
\end{bmatrix},$$

from which we can easily see that  $(0,0), (0,-\frac{1}{4}), (\frac{1}{3},0)$ are solutions. A fourth solution can be found from the linear system of equations:

$$\begin{bmatrix}
6x + 4y + 1 \\ 
3x + 8y + 1
\end{bmatrix} =
\begin{bmatrix}
0 \\ 
0
\end{bmatrix},$$ from which we can deduce that $$\begin{bmatrix}
x \\ 
y
\end{bmatrix} 
= \begin{bmatrix}
6 & 4 \\ 
3 & 8
\end{bmatrix}^{-1} \begin{bmatrix}
-1 \\ 
-1
\end{bmatrix} = 
\begin{bmatrix}
-\frac{1}{9} \\ 
-\frac{1}{12}
\end{bmatrix}, $$

and thus the fourth root $(-\frac{1}{9}, -\frac{1}{12})$.

The Hessian is given by:

$$ D^2f(x,y) = 
\begin{bmatrix}
6y & 6x + 8y + 1 \\
6x + 8y + 1 & 8x
\end{bmatrix},$$

from which we can observe

$$ D^2f(0,0) = 
\begin{bmatrix}
0 & 1 \\
1 & 0
\end{bmatrix}, \quad 
D^2f(0,-\frac{1}{4}) = 
\begin{bmatrix}
-\frac{3}{2} & -1 \\
-1 & 0
\end{bmatrix}, \quad$$
$$
D^2f(-\frac{1}{3},0) = 
\begin{bmatrix}
0 & -1 \\
-1 & -\frac{8}{3}
\end{bmatrix}, \quad
D^2f(-\frac{1}{9},-\frac{1}{12}) = 
\begin{bmatrix}
-\frac{1}{2} & -\frac{1}{3} \\
-\frac{1}{3} & -\frac{2}{3}
\end{bmatrix}$$

With mathematica (or tedious algebra), we can observe that $D^2f(-\frac{1}{9},-\frac{1}{12})$ is negative definite, so the associated root corresponds to a local maximum. The other three points are saddle points since the Hessian matrix evaluated at them is neither positive nor negative definite.

\paragraph{6.11} Consider a quadratic function $f(x) = ax^2 +bx +c$, where $a > 0$, and $b, c \in\mathbb{R}$. Show that for any initial guess $x_0 \in \mathbb{R}$, one iteration of Newton's method lands at the unique minimizer of f.

\paragraph{Solution} Note that $f(x) =  a(x + \frac{b}{2a})^2 - a(\frac{b}{2a})^2 + c$ so the minimum is clearly reached at  $x = - \frac{b}{2a}$.

Now regardless of what we choose as $x_0$, iterating one time with Newton's method $$ x_1 = x_0 - \frac{f'(x_0)}{f''(x_0)} 
= x_0 - \frac{2ax_0 + b}{2a} 
= - \frac{b}{2a}$$

\paragraph{6.14} Code up the secant method for finding a minimizer of a function. 

\paragraph{Solution} See the notebook in the same folder.


\vspace{25mm}

\bibliography{ProbStat_probset}

\end{document}